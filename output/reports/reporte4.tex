% Options for packages loaded elsewhere
% Options for packages loaded elsewhere
\PassOptionsToPackage{unicode}{hyperref}
\PassOptionsToPackage{hyphens}{url}
\PassOptionsToPackage{dvipsnames,svgnames,x11names}{xcolor}
%
\documentclass[
]{article}
\usepackage{xcolor}
\usepackage[margin=2cm]{geometry}
\usepackage{amsmath,amssymb}
\setcounter{secnumdepth}{5}
\usepackage{iftex}
\ifPDFTeX
  \usepackage[T1]{fontenc}
  \usepackage[utf8]{inputenc}
  \usepackage{textcomp} % provide euro and other symbols
\else % if luatex or xetex
  \usepackage{unicode-math} % this also loads fontspec
  \defaultfontfeatures{Scale=MatchLowercase}
  \defaultfontfeatures[\rmfamily]{Ligatures=TeX,Scale=1}
\fi
\usepackage{lmodern}
\ifPDFTeX\else
  % xetex/luatex font selection
\fi
% Use upquote if available, for straight quotes in verbatim environments
\IfFileExists{upquote.sty}{\usepackage{upquote}}{}
\IfFileExists{microtype.sty}{% use microtype if available
  \usepackage[]{microtype}
  \UseMicrotypeSet[protrusion]{basicmath} % disable protrusion for tt fonts
}{}
\makeatletter
\@ifundefined{KOMAClassName}{% if non-KOMA class
  \IfFileExists{parskip.sty}{%
    \usepackage{parskip}
  }{% else
    \setlength{\parindent}{0pt}
    \setlength{\parskip}{6pt plus 2pt minus 1pt}}
}{% if KOMA class
  \KOMAoptions{parskip=half}}
\makeatother
% Make \paragraph and \subparagraph free-standing
\makeatletter
\ifx\paragraph\undefined\else
  \let\oldparagraph\paragraph
  \renewcommand{\paragraph}{
    \@ifstar
      \xxxParagraphStar
      \xxxParagraphNoStar
  }
  \newcommand{\xxxParagraphStar}[1]{\oldparagraph*{#1}\mbox{}}
  \newcommand{\xxxParagraphNoStar}[1]{\oldparagraph{#1}\mbox{}}
\fi
\ifx\subparagraph\undefined\else
  \let\oldsubparagraph\subparagraph
  \renewcommand{\subparagraph}{
    \@ifstar
      \xxxSubParagraphStar
      \xxxSubParagraphNoStar
  }
  \newcommand{\xxxSubParagraphStar}[1]{\oldsubparagraph*{#1}\mbox{}}
  \newcommand{\xxxSubParagraphNoStar}[1]{\oldsubparagraph{#1}\mbox{}}
\fi
\makeatother


\usepackage{longtable,booktabs,array}
\usepackage{calc} % for calculating minipage widths
% Correct order of tables after \paragraph or \subparagraph
\usepackage{etoolbox}
\makeatletter
\patchcmd\longtable{\par}{\if@noskipsec\mbox{}\fi\par}{}{}
\makeatother
% Allow footnotes in longtable head/foot
\IfFileExists{footnotehyper.sty}{\usepackage{footnotehyper}}{\usepackage{footnote}}
\makesavenoteenv{longtable}
\usepackage{graphicx}
\makeatletter
\newsavebox\pandoc@box
\newcommand*\pandocbounded[1]{% scales image to fit in text height/width
  \sbox\pandoc@box{#1}%
  \Gscale@div\@tempa{\textheight}{\dimexpr\ht\pandoc@box+\dp\pandoc@box\relax}%
  \Gscale@div\@tempb{\linewidth}{\wd\pandoc@box}%
  \ifdim\@tempb\p@<\@tempa\p@\let\@tempa\@tempb\fi% select the smaller of both
  \ifdim\@tempa\p@<\p@\scalebox{\@tempa}{\usebox\pandoc@box}%
  \else\usebox{\pandoc@box}%
  \fi%
}
% Set default figure placement to htbp
\def\fps@figure{htbp}
\makeatother


% definitions for citeproc citations
\NewDocumentCommand\citeproctext{}{}
\NewDocumentCommand\citeproc{mm}{%
  \begingroup\def\citeproctext{#2}\cite{#1}\endgroup}
\makeatletter
 % allow citations to break across lines
 \let\@cite@ofmt\@firstofone
 % avoid brackets around text for \cite:
 \def\@biblabel#1{}
 \def\@cite#1#2{{#1\if@tempswa , #2\fi}}
\makeatother
\newlength{\cslhangindent}
\setlength{\cslhangindent}{1.5em}
\newlength{\csllabelwidth}
\setlength{\csllabelwidth}{3em}
\newenvironment{CSLReferences}[2] % #1 hanging-indent, #2 entry-spacing
 {\begin{list}{}{%
  \setlength{\itemindent}{0pt}
  \setlength{\leftmargin}{0pt}
  \setlength{\parsep}{0pt}
  % turn on hanging indent if param 1 is 1
  \ifodd #1
   \setlength{\leftmargin}{\cslhangindent}
   \setlength{\itemindent}{-1\cslhangindent}
  \fi
  % set entry spacing
  \setlength{\itemsep}{#2\baselineskip}}}
 {\end{list}}
\usepackage{calc}
\newcommand{\CSLBlock}[1]{\hfill\break\parbox[t]{\linewidth}{\strut\ignorespaces#1\strut}}
\newcommand{\CSLLeftMargin}[1]{\parbox[t]{\csllabelwidth}{\strut#1\strut}}
\newcommand{\CSLRightInline}[1]{\parbox[t]{\linewidth - \csllabelwidth}{\strut#1\strut}}
\newcommand{\CSLIndent}[1]{\hspace{\cslhangindent}#1}



\setlength{\emergencystretch}{3em} % prevent overfull lines

\providecommand{\tightlist}{%
  \setlength{\itemsep}{0pt}\setlength{\parskip}{0pt}}



 


\usepackage{booktabs}
\usepackage{longtable}
\usepackage{array}
\usepackage{multirow}
\usepackage{wrapfig}
\usepackage{float}
\usepackage{colortbl}
\usepackage{pdflscape}
\usepackage{tabu}
\usepackage{threeparttable}
\usepackage{threeparttablex}
\usepackage[normalem]{ulem}
\usepackage{makecell}
\usepackage{xcolor}
\usepackage[noblocks]{authblk}
\usepackage{tabu}
\usepackage{multirow}
\usepackage{booktabs}
\usepackage{pdflscape}
\renewcommand*{\Authsep}{, }
\renewcommand*{\Authand}{, }
\renewcommand*{\Authands}{, }
\renewcommand\Affilfont{\small}
\makeatletter
\@ifpackageloaded{caption}{}{\usepackage{caption}}
\AtBeginDocument{%
\ifdefined\contentsname
  \renewcommand*\contentsname{Table of contents}
\else
  \newcommand\contentsname{Table of contents}
\fi
\ifdefined\listfigurename
  \renewcommand*\listfigurename{List of Figures}
\else
  \newcommand\listfigurename{List of Figures}
\fi
\ifdefined\listtablename
  \renewcommand*\listtablename{List of Tables}
\else
  \newcommand\listtablename{List of Tables}
\fi
\ifdefined\figurename
  \renewcommand*\figurename{Figure}
\else
  \newcommand\figurename{Figure}
\fi
\ifdefined\tablename
  \renewcommand*\tablename{Table}
\else
  \newcommand\tablename{Table}
\fi
}
\@ifpackageloaded{float}{}{\usepackage{float}}
\floatstyle{ruled}
\@ifundefined{c@chapter}{\newfloat{codelisting}{h}{lop}}{\newfloat{codelisting}{h}{lop}[chapter]}
\floatname{codelisting}{Listing}
\newcommand*\listoflistings{\listof{codelisting}{List of Listings}}
\makeatother
\makeatletter
\makeatother
\makeatletter
\@ifpackageloaded{caption}{}{\usepackage{caption}}
\@ifpackageloaded{subcaption}{}{\usepackage{subcaption}}
\makeatother
\usepackage{bookmark}
\IfFileExists{xurl.sty}{\usepackage{xurl}}{} % add URL line breaks if available
\urlstyle{same}
\hypersetup{
  pdftitle={Report 4: Statistical analysis},
  pdfauthor={Andreas Laffert Tamayo},
  colorlinks=true,
  linkcolor={blue},
  filecolor={Maroon},
  citecolor={Blue},
  urlcolor={Blue},
  pdfcreator={LaTeX via pandoc}}


\title{Report 4: Statistical analysis}


  \author{Andreas Laffert Tamayo}
            \affil{%
                  Research assistant
              }
      
\date{January 30, 2026}

\usepackage{xcolor}
\definecolor{mylinkcolor}{HTML}{413395}
\hypersetup{
  colorlinks = true,
  linkcolor  = mylinkcolor,
  citecolor  = mylinkcolor,
  urlcolor   = mylinkcolor
}
\begin{document}
\maketitle


\section{Presentation}\label{presentation}

This report examines how housing wealth relates to a broad set of
social-cohesion attitudes. We conceptualize housing wealth as the market
value of the residential asset environment and proxy it using land
prices (price per square meter) in the respondent's area. The objective
is to document the magnitude and direction of these associations across
15 cohesion outcomes and to assess how they change as we sequentially
account for core socioeconomic correlates (education, social class, and
equivalised household income).

Given the panel structure of the data and relatively modest within-wave
sample sizes (2,188 person-wave observations from 821 individuals across
four waves: 2016 = 472; 2017 = 414; 2018 = 646; 2019 = 656), we estimate
pooled OLS models with wave fixed effects and cluster-robust standard
errors at the individual level. This ``pooled with time effects''
specification is a standard population-average approach: it estimates
the regression on the stacked person-wave dataset, while wave indicators
absorb common period shocks and aggregate changes over time
(\citeproc{ref-angrist_mastering_2015}{Angrist \& Pischke, 2015}).
Because repeated observations on the same individuals are typically
serially correlated, conventional i.i.d. OLS standard errors can
overstate precision; clustering at the individual level allows for
arbitrary within-person dependence in the error process, yielding valid
inference under weak conditions commonly used in applied work
(\citeproc{ref-angrist_mastering_2015}{Angrist \& Pischke, 2015}). This
approach is also well suited to an unbalanced panel (i.e., individuals
not observed in every wave), which is common in survey panel data
(\citeproc{ref-wooldridge_introductory_2009}{Wooldridge, 2009}).

The document is organized into five sequential blocks. First, we present
descriptive statistics for all variables used in the analysis. Second,
we document how the housing-wealth measure (defined as an indicator for
residing in the top quintile of land price per square meter in the
respondent's residential zone) relates to the core covariates included
in the models---age, education, social class, and equivalised household
income---and we describe its distribution across communes. Third, using
the pooled sample restricted to homeowners (fully paid-off or mortgaged;
1,391 person-wave observations from 548 individuals), we estimate pooled
OLS models for each of the 15 social-cohesion outcomes, relating the
housing-wealth indicator to each outcome. Fourth, we implement a
decomposition strategy that quantifies how much of the unconditional
(total) association between housing wealth and social cohesion is
accounted for by education, class, and income, leveraging the recursive
structure that links wealth (W) to each socioeconomic dimension (E, C,
I) and, in turn, to the cohesion outcome (Y). We report this
decomposition both for baseline specifications (without wave fixed
effects and age) and for an alternative version based on residualized
outcomes after accounting for wave fixed effects and age. Finally, we
estimate interaction models between housing wealth and homeownership
status to assess whether the association varies across tenure categories
and to explore plausible channels underlying the main results. Across
all modelling blocks, specifications are estimated sequentially,
starting with the focal housing-wealth measure plus wave fixed effects
and age, and then adding education, social class, and equivalised
household income in turn.

\section{Descriptive statistics}\label{descriptive-statistics}

First, we begin by showing the descriptive statistics for the main
variables per wave in Table~\ref{tbl-summary1}.

\begin{table}

\caption{\label{tbl-summary1}Descriptive statistics by wave}

\centering{

\begingroup\fontsize{8}{10}\selectfont

\begin{tabu} to \linewidth {>{\centering\arraybackslash}p{3cm}>{\centering\arraybackslash}p{3cm}>{\centering}X}
\toprule
Variable & Value & Overall\\
\midrule
Age &  & 44.43 (14.79)\\
\cmidrule{1-3}
Educational level &  & 12.25 (3.76)\\
\cmidrule{1-3}
Equivalised household income (square-root scale) &  & 560334.89 (685529.36)\\
\cmidrule{1-3}
 & Owned and fully paid-off home & 1031 (47.1\%)\\

 & Owned home with mortgage payments & 360 (16.5\%)\\

 & Rented housing & 493 (22.5\%)\\

\multirow[t]{-4}{*}{\centering\arraybackslash Housing tenure} & Other regime & 304 (13.9\%)\\
\cmidrule{1-3}
Housing wealth (UF, 2018) &  & 22.24 (12.72)\\
\cmidrule{1-3}
Housing wealth (log UF, 2018) &  & 2.97 (0.50)\\
\cmidrule{1-3}
 & Other quintiles & 1893 (86.5\%)\\

\multirow[t]{-2}{*}{\centering\arraybackslash Housing wealth (top 20\%)} & Top 20\% & 295 (13.5\%)\\
\cmidrule{1-3}
 & Other deciles & 2071 (94.7\%)\\

\multirow[t]{-2}{*}{\centering\arraybackslash Housing wealth (top decile)} & Decile 10 & 117 (5.3\%)\\
\cmidrule{1-3}
ISEI &  & 40.46 (16.02)\\
\cmidrule{1-3}
 & 1 & 165 (7.5\%)\\

 & 2 & 187 (8.5\%)\\

 & 3 & 215 (9.8\%)\\

 & 4 & 221 (10.1\%)\\

 & 5 & 225 (10.3\%)\\

 & 6 & 238 (10.9\%)\\

 & 7 & 239 (10.9\%)\\

 & 8 & 236 (10.8\%)\\

 & 9 & 236 (10.8\%)\\

\multirow[t]{-10}{*}{\centering\arraybackslash Income decile (equivalised)} & 10 & 226 (10.3\%)\\
\cmidrule{1-3}
 & Altruistic dispositions & 4.20 (0.61)\\

 & Conventional political participation & 1.19 (0.30)\\

 & Cultural identification & 4.12 (0.82)\\

 & Egalitarianism & 4.12 (0.78)\\

 & Generalized trust in fellow citizens & 1.31 (0.68)\\

 & Generalized trust in minorities & 2.98 (0.97)\\

 & Justification of violence & 3.94 (0.90)\\

 & Nearby network size & 3.29 (1.47)\\

 & Number of friends & 2.79 (1.18)\\

 & Political engagement & 2.12 (1.25)\\

 & Prosocial behavior & 1.74 (0.63)\\

 & Satisfaction with democracy & 2.02 (1.06)\\

 & Support for democracy & 2.45 (0.70)\\

 & Trust in major institutions & 1.72 (0.69)\\

\multirow[t]{-15}{*}{\centering\arraybackslash Social cohesion} & Unconventional political participation & 1.42 (0.66)\\
\bottomrule
\multicolumn{3}{l}{\textsuperscript{} Continuous variables report mean (SD). Categorical variables report n (\%).}\\
\end{tabu}
\endgroup{}

}

\end{table}%

\newpage{}

\section{Housing wealth quintile and
covariates}\label{housing-wealth-quintile-and-covariates}

Another important element to examine is how the wealth quintiles of
housing are distributed according to the covariates included in the
models (age, education, social class, equivalent household income, and
home ownership) and among municipalities. Below we present the means and
standard deviations of these covariates, and the frequencies and
proportions in the case of home ownership and municipalities.

\begin{figure}[H]

\caption{\label{fig-general}Homeownership by quintile housing wealth}

\centering{

\includegraphics[width=0.6\linewidth,height=\textheight,keepaspectratio]{sjtab.png}

}

\end{figure}%

\begin{table}

\caption{\label{tbl-wd-comuna}Homeownership by quintile housing wealth}

\centering{

\centering\begingroup\fontsize{9}{11}\selectfont

\begin{tabular}{>{\centering\arraybackslash}p{3cm}cccc}
\toprule
\textbf{Comune} & \textbf{N (bottom 80\%)} & \textbf{N (top 20\%)} & \textbf{\% (bottom 80\%)} & \textbf{\% (top 20\%)}\\
\midrule
Providencia & 0 & 68 & 0\% & 100\%\\
\cmidrule{1-5}
Nnunnoa & 4 & 44 & 8\% & 92\%\\
\cmidrule{1-5}
Las condes & 0 & 33 & 0\% & 100\%\\
\cmidrule{1-5}
Maipu & 152 & 17 & 90\% & 10\%\\
\cmidrule{1-5}
Puente alto & 201 & 17 & 92\% & 8\%\\
\cmidrule{1-5}
La florida & 107 & 16 & 87\% & 13\%\\
\cmidrule{1-5}
Macul & 21 & 16 & 57\% & 43\%\\
\cmidrule{1-5}
Quilicura & 32 & 16 & 67\% & 33\%\\
\cmidrule{1-5}
La reina & 15 & 14 & 52\% & 48\%\\
\cmidrule{1-5}
Pennalolen & 71 & 14 & 84\% & 16\%\\
\cmidrule{1-5}
Pudahuel & 67 & 11 & 86\% & 14\%\\
\cmidrule{1-5}
Lo barnechea & 0 & 10 & 0\% & 100\%\\
\cmidrule{1-5}
Santiago & 122 & 10 & 92\% & 8\%\\
\cmidrule{1-5}
Vitacura & 0 & 9 & 0\% & 100\%\\
\cmidrule{1-5}
Cerrillos & 44 & 0 & 100\% & 0\%\\
\cmidrule{1-5}
Cerro navia & 50 & 0 & 100\% & 0\%\\
\cmidrule{1-5}
Conchali & 58 & 0 & 100\% & 0\%\\
\cmidrule{1-5}
El bosque & 80 & 0 & 100\% & 0\%\\
\cmidrule{1-5}
Estacion central & 52 & 0 & 100\% & 0\%\\
\cmidrule{1-5}
Huechuraba & 61 & 0 & 100\% & 0\%\\
\cmidrule{1-5}
Independencia & 30 & 0 & 100\% & 0\%\\
\cmidrule{1-5}
La cisterna & 40 & 0 & 100\% & 0\%\\
\cmidrule{1-5}
La granja & 36 & 0 & 100\% & 0\%\\
\cmidrule{1-5}
La pintana & 104 & 0 & 100\% & 0\%\\
\cmidrule{1-5}
Lo espejo & 46 & 0 & 100\% & 0\%\\
\cmidrule{1-5}
Lo prado & 26 & 0 & 100\% & 0\%\\
\cmidrule{1-5}
Padre hurtado & 27 & 0 & 100\% & 0\%\\
\cmidrule{1-5}
Pedro aguirre cerda & 61 & 0 & 100\% & 0\%\\
\cmidrule{1-5}
Quinta normal & 34 & 0 & 100\% & 0\%\\
\cmidrule{1-5}
Recoleta & 56 & 0 & 100\% & 0\%\\
\cmidrule{1-5}
Renca & 52 & 0 & 100\% & 0\%\\
\cmidrule{1-5}
San bernardo & 118 & 0 & 100\% & 0\%\\
\cmidrule{1-5}
San joaquin & 44 & 0 & 100\% & 0\%\\
\cmidrule{1-5}
San miguel & 29 & 0 & 100\% & 0\%\\
\cmidrule{1-5}
San ramon & 53 & 0 & 100\% & 0\%\\
\bottomrule
\end{tabular}
\endgroup{}

}

\end{table}%

\begin{table}

\caption{\label{tbl-wd-age}Quintile housing wealth by age, education,
social class and household income}

\centering{

\centering\begingroup\fontsize{9}{11}\selectfont

\begin{tabular}{>{\centering\arraybackslash}p{3cm}cccc}
\toprule
\textbf{Quintile housing wealth} & \textbf{Age} & \textbf{Education} & \textbf{ISEI} & \textbf{Equivalised household income}\\
\midrule
Q1 & 43.8 (16.3) & 11.5 (3.4) & 36.4 (13) & \$  388,870 (\$  344,118)\\
\cmidrule{1-5}
Q2 & 45.1 (14.3) & 11.1 (3.9) & 36.7 (14.2) & \$  419,406 (\$  442,975)\\
\cmidrule{1-5}
Q3 & 44 (14.6) & 11.8 (3.4) & 38.1 (15.2) & \$  469,003 (\$  659,992)\\
\cmidrule{1-5}
Q4 & 45.2 (14.7) & 13.1 (3.3) & 41.4 (15.2) & \$  555,507 (\$  498,900)\\
\cmidrule{1-5}
Q5 & 43.5 (14.3) & 15.2 (3) & 56 (16.1) & \$1,231,494 (\$1,149,961)\\
\bottomrule
\end{tabular}
\endgroup{}

}

\end{table}%

\newpage{}

\section{Primary set: Social cohesion and housing-wealth extremes
(top-quintile land-price
exposure)}\label{primary-set-social-cohesion-and-housing-wealth-extremes-top-quintile-land-price-exposure}

We then focus on the upper tail of the local housing-wealth distribution
by using an indicator of exposure to extreme housing-wealth contexts.
Specifically, we construct a dummy variable equal to 1 if the respondent
resides in a zone where land price per square meter falls in the top
quintile (top 20\%), and 0 otherwise. This specification is designed to
capture potentially discontinuous differences between living in the most
expensive residential contexts and living elsewhere in the
distribution---an ``extremes'' contrast that may be missed by a linear
specification in log prices. To align this measure with housing wealth
as an owned asset, we estimate these models on a restricted panel of
respondents who report owning their dwelling, either fully paid-off or
with an outstanding mortgage (based on the homeownership item). Models
are estimated using pooled OLS with wave fixed effects and age, CR2
individual-clustered standard errors, and sequential adjustment for
education, social class, and equivalised household income.

Formally, the model is:

\begin{equation}\phantomsection\label{eq-third}{
Y_{it}=\alpha+\beta_{1}\text{TopQuintilePrice}_{it}+\beta{2}\text{Education}_{it}+\beta{3}\text{Class}_{it}+\beta_{4}\text{Income}_{it}+\lambda_{t}+\delta_{it}+\varepsilon_{it}
}\end{equation}

where \(Y_{it}\) is the social-cohesion outcome for individual \(i\) in
wave \(t\); \(\alpha\) is the intercept; \(\beta_{1}\) captures the
association between residing in a top-quintile land-price zone (top 20\%
vs.~all others) and \(Y_{it}\), conditional on the covariates;
\(\beta_{2}\), \(\beta_{3}\), and \(\beta_{4}\) capture the associations
of years of education, the International Socio-Economic Index of
Occupational Status (ISEI), and the log equivalised household income
respectively; \(\lambda_{t}\) denotes wave fixed effects,
\(\delta_{it}\) denotes individual's age, and \(\varepsilon_{it}\) is
the idiosyncratic error term. Standard errors are clustered at the
respondent level (\texttt{idencuesta}) using the CR2 correction.

Below, we present the results of the estimates using coefficient plots
grouped by indicators belonging to the cultural, relational, political,
and normative dimensions of social cohesion
(\citeproc{ref-otero_lives_2022}{Otero et al., 2022}). Complete tables
can be found in Supplementary Material.

\subsection{Cultural dimension}\label{cultural-dimension}

\begin{figure}[H]

\centering{

\includegraphics[width=0.9\linewidth,height=\textheight,keepaspectratio]{reporte4_files/figure-pdf/fig-cultural-1.pdf}

}

\caption{\label{fig-cultural}Cultural identification by top-quintile
housing wealth, educational level, social class and income}

\end{figure}%

\subsection{Relational dimension}\label{relational-dimension}

\begin{figure}[H]

\centering{

\includegraphics[width=0.9\linewidth,height=\textheight,keepaspectratio]{reporte4_files/figure-pdf/fig-relational-1.pdf}

}

\caption{\label{fig-relational}Relational dimension of social cohesion
by top-quintile housing wealth, educational level, social class and
income}

\end{figure}%

\subsection{Political dimension}\label{political-dimension}

\begin{figure}[H]

\centering{

\includegraphics[width=0.9\linewidth,height=\textheight,keepaspectratio]{reporte4_files/figure-pdf/fig-political-1.pdf}

}

\caption{\label{fig-political}Political dimension of social cohesion by
top-quintile housing wealth, educational level, social class and income}

\end{figure}%

\subsection{Normative dimension}\label{normative-dimension}

\begin{figure}[H]

\centering{

\includegraphics[width=0.9\linewidth,height=\textheight,keepaspectratio]{reporte4_files/figure-pdf/fig-normative-1.pdf}

}

\caption{\label{fig-normative}Normative dimension of social cohesion by
top-quintile housing wealth, educational level, social class and income}

\end{figure}%

\newpage{}

\section{Decomposition}\label{decomposition}

\subsection{Baseline decomposition}\label{baseline-decomposition}

The baseline decomposition partitions the total association between
housing wealth and each social-cohesion outcome into a residual (direct)
component and three mediated components, following the linear system
presented in the Social cohesion and housing wealth in Chile document.
Let \(Y\) denote a measure of social cohesion and \(W\) housing wealth.
The total association is obtained from the unconditional model:

\[
Y = \alpha_0 + \beta_0 W + u_0 
\]

where \(\beta_0\) captures the overall (unconditional) association
between \(W\) and \(Y\). A second model introduces socioeconomic
mediators---education \(E\), social class \(C\) (ISEI), and household
income \(I\):

\[
Y = \alpha_1 + \beta_1 W + \gamma_1 E + \theta_1 C + \delta_1 I + u_1 
\]

In this specification, \(\beta_1\) represents the residual association
between \(W\) and \(Y\) net of \(E\), \(C\), and \(I\), while
\(\gamma_1\), \(\theta_1\), and \(\delta_1\) are the conditional
associations of each mediator with \(Y\). To quantify mediated pathways,
each mediator is regressed on \(W\) in separate first-stage equations:

\[
\begin{aligned}
E &= \alpha_E + \gamma_2 W + u_E \\
C &= \alpha_C + \theta_2 W + u_C \\
I &= \alpha_I + \delta_2 W + u_I
\end{aligned}
\]

Under the linear decomposition, the total association \(\beta_0\) can be
expressed as the sum of the residual component and the three indirect
components:

\[
\beta_0 = \beta_1 + (\gamma_1\gamma_2) + (\theta_1\theta_2) + (\delta_1\delta_2)
\]

Accordingly, the education-mediated contribution is computed as
\(\gamma_1\gamma_2\), the class-mediated contribution as
\(\theta_1\theta_2\), and the income-mediated contribution as
\(\delta_1\delta_2\). Results are additionally summarized using
component shares---e.g., \(\beta_1/\beta_0\) and
\((\gamma_1\gamma_2)/\beta_0\)---to express each pathway as a proportion
of the total association.

In the accompanying implementation, regressions used to estimate
decomposition parameters are estimated with CR2 cluster-robust standard
errors at the individual level, and uncertainty for derived components
(products and shares) is obtained via cluster bootstrap procedures.

\begin{table}

\caption{\label{tbl-decomp-base}Baseline decomposition of the
housing-wealth association: indirect components via education, social
class (ISEI), and income}

\centering{

\begingroup\fontsize{7}{9}\selectfont

\begin{tabu} to \linewidth {>{\centering\arraybackslash}p{4cm}>{\centering}X>{\centering}X>{\centering}X>{\centering}X}
\toprule
\textbf{Outcome} & \textbf{B1 Residual} & \textbf{E\%} & \textbf{C\%} & \textbf{I\%}\\
\midrule
Cultural identification & 0.786 (0.388, 1.032) & 0.196 (-0.064, 0.584) & 0.12 (-0.136, 0.508) & -0.102 (-0.503, 0.162)\\
Number of friends & 0.22 (-0.353, 0.433) & 0.476 (0.295, 0.808) & 0.154 (-0.066, 0.404) & 0.151 (0.005, 0.382)\\
Network size & 0.571 (0.181, 0.804) & 0.235 (0.091, 0.448) & 0.3 (0.096, 0.638) & -0.107 (-0.255, 0.061)\\
Generalized trust & 0.11 (-1.196, 0.624) & 0.395 (0.13, 1.278) & 0.282 (-0.1, 1.001) & 0.213 (-0.152, 0.966)\\
Trust in minorities & 0.339 (-0.245, 0.609) & 0.56 (0.339, 1.162) & 0.026 (-0.26, 0.297) & 0.075 (-0.104, 0.281)\\
\addlinespace
Trust in major institutions & -0.017 (-3.118, 0.643) & 0.695 (0.266, 3.436) & 0.176 (-0.356, 1.305) & 0.146 (-0.341, 0.91)\\
Political engagement & 0.108 (-0.463, 0.431) & 0.554 (0.345, 0.942) & 0.229 (0.053, 0.483) & 0.108 (-0.032, 0.355)\\
Satisfaction with democracy & -0.961 (-4.872, 16.585) & 0.401 (-6.153, 2.018) & 0.938 (-4.944, 5.134) & 0.622 (-4.874, 3.653)\\
Conventional political participation & -0.141 (-3.965, 7.654) & 0.456 (-1.367, 2.494) & 0.13 (-1.557, 1.411) & 0.555 (-3.204, 2.431)\\
Unconventional political participation & -0.145 (-3.312, 0.364) & 1.018 (0.497, 4.516) & 0.254 (0.019, 1.386) & -0.127 (-0.827, 0.152)\\
\addlinespace
Egalitarianism & 0.761 (0.261, 1.034) & -0.013 (-0.316, 0.129) & -0.039 (-0.431, 0.317) & 0.292 (0.026, 0.855)\\
Altruistic dispositions & 1.072 (-1.524, 1.871) & 0.927 (-9.367, 4.834) & -0.11 (-2.211, 2.504) & -0.888 (-4.193, 10.708)\\
Prosocial behavior & 0.096 (-5.591, 0.594) & 0.372 (0.098, 2.272) & 0.059 (-0.468, 0.711) & 0.472 (0.063, 2.628)\\
Democracy support & -0.953 (-17.929, 15.475) & 1.183 (-9.792, 11.083) & 0.558 (-3.547, 5.222) & 0.212 (-1.758, 2.742)\\
Justification of violence & 0.5 (-2.816, 3.733) & 0.213 (-1.156, 1.618) & 0.475 (-1.955, 3.966) & -0.187 (-2.566, 0.915)\\
\bottomrule
\multicolumn{5}{l}{\rule{0pt}{1em}\textit{Note: }}\\
\multicolumn{5}{l}{\rule{0pt}{1em}W = Housing Wealth, E = Education, C = Social Class, I = Household Income.}\\
\multicolumn{5}{l}{\rule{0pt}{1em}Coefficient entries report indirect components (products).}\\
\multicolumn{5}{l}{\rule{0pt}{1em}Shares are computed as component/B0, where B0 is the total association from the unconditional model Y \textasciitilde{} W.}\\
\multicolumn{5}{l}{\rule{0pt}{1em}95\% CIs are obtained via cluster bootstrap (999 iterations) resampling individuals and reported in parenthesis.}\\
\multicolumn{5}{l}{\rule{0pt}{1em}Shares may be negative or exceed 1 when components and β0 have opposite signs or when B0 is close to zero.}\\
\end{tabu}
\endgroup{}

}

\end{table}%

\newpage{}

\subsection{Residualized
decomposition}\label{residualized-decomposition}

The residualized decomposition implements the same decomposition after
partialling out wave fixed effects and age, thereby incorporating these
adjustments while preserving the tractable three-mediator system.
Residualized variables \({Y}, {W}, {E}, {C}, {I}\) are obtained by
regressing each variable on wave indicators \(\Lambda_t\) and age \(A\)
and retaining the residuals; for example:

\[
Y = a_Y + \Lambda_t + \phi A + e_Y \quad \Rightarrow \quad {Y} = e_Y
\]

with analogous residualization for \(W\) and each mediator. The baseline
system is then estimated using residualized quantities:

\[
{Y} = {\alpha}_0 + {\beta}_0 {W} + {u}_0
\]

\[
{Y} = {\alpha}_1 + {\beta}_1 {W} + {\gamma}_1 {E} + {\theta}_1 {C} + {\delta}_1 {I}+ {u}_1
\]

\[
\begin{aligned}
{E} = {\alpha}_E + {\gamma}_2 {W} + {u}_E \\
{C} = {\alpha}_C + {\theta}_2 {W} + {u}_C \\
{I} = {\alpha}_I + {\delta}_2 {W} + {u}_I 
\end{aligned}
\]

The decomposition identity is preserved on the residual scale:

\[
{\beta}_0 = {\beta}_1 + ({\gamma}_1{\gamma}_2) + ({\theta}_1{\theta}_2) + ({\delta}_1{\delta}_2)
\]

so components are interpreted as associations between housing wealth and
social cohesion after removing variation attributable to age and wave
membership. In the accompanying implementation, regressions used to
estimate decomposition parameters are estimated with CR2 cluster-robust
standard errors at the individual level, and uncertainty for derived
components (products and shares) is obtained via cluster bootstrap
procedures.

\begin{table}

\caption{\label{tbl-decomp-resid}Residualized decomposition of the
housing-wealth association: indirect components via education, social
class (ISEI), and income}

\centering{

\begingroup\fontsize{7}{9}\selectfont

\begin{tabu} to \linewidth {>{\centering\arraybackslash}p{4cm}>{\centering}X>{\centering}X>{\centering}X>{\centering}X}
\toprule
\textbf{Outcome} & \textbf{B1 Residual} & \textbf{E\%} & \textbf{C\%} & \textbf{I\%}\\
\midrule
Cultural identification & 0.864 (0.51, 1.13) & -0.002 (-0.272, 0.302) & 0.148 (-0.087, 0.636) & -0.01 (-0.334, 0.256)\\
Number of friends & 0.274 (-0.244, 0.489) & 0.385 (0.209, 0.635) & 0.162 (-0.055, 0.42) & 0.179 (0.035, 0.387)\\
Network size & 0.592 (0.193, 0.811) & 0.247 (0.074, 0.473) & 0.286 (0.079, 0.619) & -0.125 (-0.276, 0.048)\\
Generalized trust & 0.14 (-1.367, 0.645) & 0.366 (0.039, 1.414) & 0.278 (-0.096, 1.014) & 0.217 (-0.2, 0.968)\\
Trust in minorities & 0.397 (-0.159, 0.66) & 0.482 (0.241, 1.063) & 0.026 (-0.256, 0.304) & 0.095 (-0.075, 0.316)\\
\addlinespace
Trust in major institutions & -0.028 (-2.884, 0.643) & 0.745 (0.284, 3.618) & 0.183 (-0.365, 1.32) & 0.1 (-0.469, 0.953)\\
Political engagement & 0.095 (-0.506, 0.435) & 0.518 (0.31, 0.886) & 0.262 (0.085, 0.522) & 0.124 (-0.021, 0.312)\\
Satisfaction with democracy & -0.978 (-4.655, 24.863) & 0.786 (-11.984, 8.791) & 0.872 (-11.248, 2.488) & 0.32 (-7.309, 6.489)\\
Conventional political participation & -0.292 (-4.35, 8.284) & 0.655 (-2.534, 3.32) & 0.136 (-1.439, 1.393) & 0.501 (-2.619, 1.908)\\
Unconventional political participation & -0.041 (-2.295, 0.499) & 0.822 (0.29, 2.871) & 0.265 (-0.014, 1.314) & -0.046 (-0.544, 0.422)\\
\addlinespace
Egalitarianism & 0.701 (0.128, 0.972) & 0.048 (-0.18, 0.239) & -0.04 (-0.417, 0.296) & 0.29 (0.022, 0.825)\\
Altruistic dispositions & 1.005 (-0.972, 2.304) & 0.664 (-4.199, 4.195) & 0.086 (-1, 4.031) & -0.755 (-5.462, 4.542)\\
Prosocial behavior & 0.091 (-5.061, 0.593) & 0.374 (0.069, 2.11) & 0.073 (-0.46, 0.721) & 0.463 (0.049, 2.511)\\
Democracy support & -1.125 (-17.589, 21.165) & 1.193 (-12.435, 11.213) & 0.652 (-5.864, 5.211) & 0.281 (-2.632, 3.686)\\
Justification of violence & 0.526 (-3.313, 6.434) & 0.275 (-0.411, 2.594) & 0.417 (-1.628, 4.96) & -0.217 (-4.194, 0.957)\\
\bottomrule
\multicolumn{5}{l}{\rule{0pt}{1em}\textit{Note: }}\\
\multicolumn{5}{l}{\rule{0pt}{1em}W = Housing Wealth, E = Education, C = Social Class, I = Household Income.}\\
\multicolumn{5}{l}{\rule{0pt}{1em}All variables are residualized with respect to wave fixed effects and age prior to estimation (i.e., net of wave and age).}\\
\multicolumn{5}{l}{\rule{0pt}{1em}Coefficient entries report indirect components (products) estimated on residualized variables.}\\
\multicolumn{5}{l}{\rule{0pt}{1em}Shares are computed as component/B0, where β0 is the total association from the unconditional residualized model (Y \textasciitilde{} W).}\\
\multicolumn{5}{l}{\rule{0pt}{1em}95\% CIs are obtained via cluster bootstrap (999 iterations) resampling individuals and reported in parentheses.}\\
\multicolumn{5}{l}{\rule{0pt}{1em}Shares may be negative or exceed 1 when components and β0 have opposite signs or when B0 is close to zero.}\\
\end{tabu}
\endgroup{}

}

\end{table}%

\newpage{}

\section{Interactions: Social cohesion and housing-wealth extremes
(top-quintile land-price exposure) by
homeownership}\label{interactions-social-cohesion-and-housing-wealth-extremes-top-quintile-land-price-exposure-by-homeownership}

We next assess whether the association between residing in a top
housing-wealth context---defined as living in a zone within the top
quintile (top 20\%) of land price per square meter---and social-cohesion
outcomes varies by homeownership status. To this end, we estimate models
on a restricted panel of homeowners, distinguishing between respondents
who own their dwelling outright and those who own with an outstanding
mortgage.

Formally, the model is:

\begin{equation}\phantomsection\label{eq-for}{
\begin{aligned}
Y_{it} &=
\sum_{g=1}^{4}\alpha_g\,H_{git}
\;+\;
\sum_{g=1}^{4}\beta_g\Big(\text{Top Quintile Price}_{it}\times H_{git}\Big)
\;+\;
\gamma_1\,\text{Education}_{it}
\;+\;
\gamma_2\,\text{Class}_{it}
\;+\;
\gamma_3\,\text{Income}_{it} \\
&\quad + \delta\,\text{Age}_{it}
\;+\;
\lambda_t
\;+\;
\varepsilon_{it}
\end{aligned}
}\end{equation}

where \(H_{git}\) is a dummy indicating whether individual \(i\) in wave
\(t\) belongs to homeownership category \(g \in \{1,2,3,4\}\). The
coefficients \(\beta_g\) represent the top-quintile housing wealth (top
20\% 10 vs.~all others) gradient within each homeownership category
(i.e., group-specific slopes), and there is no global \(\beta_1\) term
for \(\text{Top top 10 decile Price}_{it}\) outside these interactions;
\(\beta_{2}\), \(\beta_{3}\), and \(\beta_{4}\) capture the associations
of years of education, the International Socio-Economic Index of
Occupational Status (ISEI), and the log equivalised household income
respectively; and \(\beta_{6}\) captures the interaction between
top-quintile housing wealth and homeownership status, indicating whether
the top 20\% of housing wealth effect differs across ownership groups.
\(\lambda_{t}\) denotes wave fixed effects, \(\delta_{it}\) denotes
individual's age, and \(\varepsilon_{it}\) is the idiosyncratic error
term. Standard errors are clustered at the respondent level
(\texttt{idencuesta}) using the CR2 correction.

Below, we present the results of the estimates using plots of predicted
values grouped by indicators belonging to the cultural, relational,
political, and normative dimensions of social cohesion. Complete tables
can be found in Supplementary Material.

\subsection{Cultural dimension}\label{cultural-dimension-1}

\begin{figure}[H]

{\centering \includegraphics[width=0.8\linewidth,height=\textheight,keepaspectratio]{reporte4_files/figure-pdf/unnamed-chunk-2-1.pdf}

}

\caption{Cultural identification by top 20\% housing wealth and
homeownership (controls: education, social class, equivalised income,
age)}

\end{figure}%

\subsection{Relational dimension}\label{relational-dimension-1}

\begin{figure}[H]

{\centering \includegraphics[width=0.8\linewidth,height=\textheight,keepaspectratio]{reporte4_files/figure-pdf/unnamed-chunk-3-1.pdf}

}

\caption{Number of friends by top 20\% housing wealth and homeownership
(controls: education, social class, equivalised income, age)}

\end{figure}%

\begin{figure}[H]

{\centering \includegraphics[width=0.8\linewidth,height=\textheight,keepaspectratio]{reporte4_files/figure-pdf/unnamed-chunk-4-1.pdf}

}

\caption{Network size by top 20\% housing wealth and homeownership
(controls: education, social class, equivalised income, age)}

\end{figure}%

\begin{figure}[H]

{\centering \includegraphics[width=0.8\linewidth,height=\textheight,keepaspectratio]{reporte4_files/figure-pdf/unnamed-chunk-5-1.pdf}

}

\caption{Generalized trust by top 20\% housing wealth and homeownership
(controls: education, social class, equivalised income, age)}

\end{figure}%

\begin{figure}[H]

{\centering \includegraphics[width=0.8\linewidth,height=\textheight,keepaspectratio]{reporte4_files/figure-pdf/unnamed-chunk-6-1.pdf}

}

\caption{Trust in minorities by top 20\% housing wealth and
homeownership (controls: education, social class, equivalised income,
age)}

\end{figure}%

\begin{figure}[H]

{\centering \includegraphics[width=0.8\linewidth,height=\textheight,keepaspectratio]{reporte4_files/figure-pdf/unnamed-chunk-7-1.pdf}

}

\caption{Trust in major institutions by top 20\% housing wealth and
homeownership (controls: education, social class, equivalised income,
age)}

\end{figure}%

\subsection{Political dimension}\label{political-dimension-1}

\begin{figure}[H]

{\centering \includegraphics[width=0.8\linewidth,height=\textheight,keepaspectratio]{reporte4_files/figure-pdf/unnamed-chunk-8-1.pdf}

}

\caption{Political engagement by top 20\% housing wealth and
homeownership (controls: education, social class, equivalised income,
age)}

\end{figure}%

\begin{figure}[H]

{\centering \includegraphics[width=0.8\linewidth,height=\textheight,keepaspectratio]{reporte4_files/figure-pdf/unnamed-chunk-9-1.pdf}

}

\caption{Satisfaction with democracy by top 20\% housing wealth and
homeownership (controls: education, social class, equivalised income,
age)}

\end{figure}%

\begin{figure}[H]

{\centering \includegraphics[width=0.8\linewidth,height=\textheight,keepaspectratio]{reporte4_files/figure-pdf/unnamed-chunk-10-1.pdf}

}

\caption{Conventional political participation by top 20\% housing wealth
and homeownership (controls: education, social class, equivalised
income, age)}

\end{figure}%

\begin{figure}[H]

{\centering \includegraphics[width=0.8\linewidth,height=\textheight,keepaspectratio]{reporte4_files/figure-pdf/unnamed-chunk-11-1.pdf}

}

\caption{Unconventional political participation by top 20\% housing
wealth and homeownership (controls: education, social class, equivalised
income, age)}

\end{figure}%

\begin{figure}[H]

{\centering \includegraphics[width=0.8\linewidth,height=\textheight,keepaspectratio]{reporte4_files/figure-pdf/unnamed-chunk-12-1.pdf}

}

\caption{Egalitarianism by top 20\% housing wealth and homeownership
(controls: education, social class, equivalised income, age)}

\end{figure}%

\begin{figure}[H]

{\centering \includegraphics[width=0.8\linewidth,height=\textheight,keepaspectratio]{reporte4_files/figure-pdf/unnamed-chunk-13-1.pdf}

}

\caption{Altruistic dispositions by top 20\% housing wealth and
homeownership (controls: education, social class, equivalised income,
age)}

\end{figure}%

\begin{figure}[H]

{\centering \includegraphics[width=0.8\linewidth,height=\textheight,keepaspectratio]{reporte4_files/figure-pdf/unnamed-chunk-14-1.pdf}

}

\caption{Prosocial behavior by top 20\% housing wealth and homeownership
(controls: education, social class, equivalised income, age)}

\end{figure}%

\subsection{Normative dimension}\label{normative-dimension-1}

\begin{figure}[H]

{\centering \includegraphics[width=0.8\linewidth,height=\textheight,keepaspectratio]{reporte4_files/figure-pdf/unnamed-chunk-15-1.pdf}

}

\caption{Democracy support by top 20\% housing wealth and homeownership
(controls: education, social class, equivalised income, age)}

\end{figure}%

\begin{figure}[H]

{\centering \includegraphics[width=0.8\linewidth,height=\textheight,keepaspectratio]{reporte4_files/figure-pdf/unnamed-chunk-16-1.pdf}

}

\caption{Justification of violence by top 20\% housing wealth and
homeownership (controls: education, social class, equivalised income,
age)}

\end{figure}%

\section{References}\label{references}

\phantomsection\label{refs}
\begin{CSLReferences}{1}{0}
\bibitem[\citeproctext]{ref-angrist_mastering_2015}
Angrist, J. D., \& Pischke, J.-S. (2015). \emph{Mastering 'metrics: The
path from cause to effect}. Princeton, NJ Oxford: Princeton University
Press.

\bibitem[\citeproctext]{ref-otero_lives_2022}
Otero, G., Volker, B., Rözer, J., \& Mollenhorst, G. (2022). The lives
of others: {Class} divisions, network segregation, and attachment to
society in {Chile}. \emph{The British Journal of Sociology},
\emph{73}(4), 754--785. \url{https://doi.org/10.1111/1468-4446.12966}

\bibitem[\citeproctext]{ref-wooldridge_introductory_2009}
Wooldridge, J. M. (2009). \emph{{Introductory econometrics: a modern
approach}} (4th ed). Mason, OH: South Western, Cengage Learning.

\end{CSLReferences}




\end{document}
